        \title[\whatshort, \simplenum, 
Slide \insertframenumber/\inserttotalframenumber ] {\what}
% this dirty hack allows me to display frame numbers in the footnotebar.

 

\subtitle{Class \simplenum: Reinforcement Learning}



\usepackage{graphicx}
%\subtitle{}

\begin{document}

\begin{frame}
  \titlepage

\end{frame}

\begin{frame}
  \frametitle{Class Outline} \tableofcontents % You might wish to add
  %the option [pausesections]
\end{frame}


% Since this a solution template for a generic talk, very little can
% be said about how it should be structured. However, the talk length
% of between 15min and 45min and the theme suggest that you stick to
% the following rules:  

% - Exactly two or three sections (other than the summary).
% - At *most* three subsections per section.
% - Talk about 30s to 2min per frame. So there should be between about
%   15 and 30 frames, all told.

\begin{frame}[fragile]
\frametitle{References}

\begin{itemize}

\item[[SB{]}] R.~S.~Sutton and A.~G.~Barto, 
``Reinforcement Learning: An Introduction'', 2nd edition, The MIT
Press, 2018

\item[[CS{]}]  C.~Szepesva\'ri
``Algorithms for Reinforcement Learning'', Morgan \& Claypool, 2010

\item[[JT{]}] J.~N.~Tsitsiklis, On the Convergence of Optimistic Policy
Iteration, JMLR 3 (2002) 59-72

\item[[WD{]}] C.~J.~C.~H.~Watkins and P.~Dayan, Technical Note:
Q-Learning, Machine Learning, 8, 279-292 (1992)

\end{itemize}
\end{frame}

\section[Preliminaries]{Motivation and Preliminaries}

\begin{frame}\frametitle{Example: USPS}

\begin{itemize}

\item the problem is to label an image, which is a $16\times16$ matrix of pixels

--- it is known that an image represents a hand-written
digit, from 0 to 9

\item we are given a training set containing a large
number of labelled images

--- USPS dataset: scanned zip codes from envelopes

\end{itemize}
\end{frame}

\end{document}










